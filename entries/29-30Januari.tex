%!TEX root = ../main.tex

\chapter{Progres Belajar}
\section{Belajar materi pendidikan dan latex}
Kegiatan-kegiatan yang tercantum pada bab ini saya laksanakan pada tanggal 29 hingga 30 Januari 2026.

\subsection{Aktivitas Hari Ini}
Hari ini rencananya mau bikin rangkuman, tapi malah berakhir \textit{debugging} environment seharian. Pertama kali belajar nulis latex dari vscode, ternyata banyak hal yang perlu diinstal terus debugging dan troubleshoot banyak hal. Untuk materi \textit{day} 1 dan 2 cuma bisa dibaca-baca ulang, untuk besok pagi akan segera dipahami dan akan dibuat rangkumannya.

\vspace{1cm}
Hal-hal yang berhasil dikerjakan:
\begin{itemize}
    \item Membaca ulang materi pendidikan serta memahami.
    \item Inisialisasi Git dan menghubungkan remote ke GitHub.
    \item Install ekstensi \textit{LaTeX Workshop} di VS Code.
    \item Install compiler MiKTeX (setelah sempat diblokir Windows Defender).
    \item Troubleshooting error build beruntun.
\end{itemize}

\subsection{Log Error \& Solusi (The Real Struggle)}
Sebenernya error lama ini mayoritas disebabkan karena pengen liat preview di vscode. Meski lama ini merupakan investasi buat kedepannya saat ngga make github. 

\subsubsection{1. Masalah Instalasi Compiler}
Awalnya preview tidak muncul sama sekali (abu-abu). Ternyata karena belum ada \textit{compiler}. Saat install MiKTeX, sempat tertahan layar ungu \textit{"Windows protected your PC"}.
\textbf{Solusi:} Paksa jalan dengan klik \textit{Run anyway}.

\subsubsection{2. Syntax Error: Simbol Dan (\&)}
Build gagal total dengan pesan error yang membingungkan:
\begin{lstlisting}[language=bash, caption=Error Simbol]
! Misplaced alignment tab character &.
\end{lstlisting}
\textbf{Analisis:} Ternyata di LaTeX, simbol `\&` dipakai untuk membuat tabel. 
\textbf{Fix:} Harus menulisnya dengan \textit{backslash} di depannya (\texttt{\textbackslash\&}). Bonus: waktu nulis ini sampe selesai si "\&" masih eror lagi jir sampe agak lama baru notice...

\subsubsection{3. VS Code vs Terminal}
VS Code terus-menerus memberikan notifikasi \textit{"Recipe terminated with error"} dan mengeluh soal update MiKTeX, padahal file PDF sebenarnya berhasil dibuat.
\textbf{Workaround:} Saya mem-bypass tombol build VS Code dan menggunakan terminal manual:
\begin{lstlisting}[language=bash]
pdflatex main.tex
\end{lstlisting}
Hasilnya: \texttt{Output written on main.pdf}. Namun, sekarang udah bener-bener bisa setelah instal "pearl" (jujur kurang tau gunanya, next cari tau)

